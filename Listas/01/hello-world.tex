\documentclass{article}
\usepackage{hyperref}
\usepackage[shortlabels]{enumitem}
\usepackage{geometry}
\geometry{a4paper, left=3cm, right=2cm, top=3cm,
bottom=2cm}

\begin{document}
      \begin{enumerate}
        \item \textbf{Realize as seguintes mudanças de bases numéricas}
        \begin{enumerate}[(a)]
          \item \textbf{ 213 para a base 2}
            \begin{math}
              \\ 213_{10} = (1 \cdot 2^{7}) \ + \ (1 \cdot 2^{6}) \ + \ (1 \cdot 2^{4}) \ + \ (1 \cdot 2^{2}) \ + \ (1 \cdot 2^{0})
              \\ 213_{10} = 11010101_{2}
            \end{math}
          \item \textbf{213 para a base 3}
            \begin{math}
              \\ 213_{10} = (2 \cdot 3^{4}) \ + \ (1 \cdot 3^{3}) \ + \ (2 \cdot 3^{2}) \ + \ (2 \cdot 3^{1})
              \\ 213_{10} = 21220_{3}
            \end{math}
          \item \textbf{197 para a base 2}
            \begin{math}
              \\ 197_{10} = (1 \cdot 2^{7}) \ + \ (1 \cdot 2^{6}) \ + \ (1 \cdot 2 ^{2}) \ + \ (1 \cdot 2^{0})
              \\ 197_{10} = 11000101_{2}
            \end{math}
          \item \textbf{197 para a base 4}
            \begin{math}
              \\ 197_{4} = (3 \cdot 4^{3}) \ + \ (1 \cdot 2^{1}) \ + \ (1 \cdot 2^{0})
              \\ 197_{4} = 3011_{4}
            \end{math}
          \item \textbf{197 para a base hexadecimal}
            \begin{math}
              \\ 197_{10} = (12 \cdot 16^{1}) \ + \ (5 \cdot 16^{0})
              \\ 197_{10} = C5_{16}
            \end{math}
        \end{enumerate}
        \item \textbf{Resolva as seguintes questões:}
        \begin{enumerate}[(a)]
          \item \textbf{Calcule $ 010101_{2} \ + \ 001101_{2} $ e confira o resultado na base decimal.}
          \begin{math}
            \\ 010101_{2} \ + \ 001101_{2} = 100010_{2}
            \\ 21_{10} \ \ \ \ \ \  + \ \ \ \ \ 13_{10} \  = \ \ 34_{10}
          \end{math}
          \item \textbf{Tome os resultados dos itens (a) e (c) da questão anterior. Subtraia o primeiro do segundo.}
          \begin{math}
            \\ 11010101_{2} \ - \
               11000101_{2} = 00010000_{2}
          \end{math}
          \item \textbf{Quanto é $ 1102_{3} + 0211_{3}$?}
          \begin{math}
            \\ 1102_{3} \ + \
               0211_{3} = 2020_{3}
          \end{math}
          \item \textbf{Qual o resultado do item anterior em decimal?}
          \begin{math}
            \\ 2020_{3} = (2 \cdot 3^{3}) \ + \ (2 \cdot 3^{1})
            \\ 2020_{3} = 54_{10} + 6_{10} = 60_{10}
          \end{math}
        \end{enumerate}
        \item \textbf{Resolva as seguintes questões usando suas proprias palavras}
        \begin{enumerate}
          \item \textbf{O que é a representação binária por complemento de 2? Para que serve?}
          \linebreak É um técnica que se usa para representar números negativos na base binária, onde para representarmos um número negativo precisamos inverter os bits do número escolhido e somar 1, assim, através do bit mais significativo vamos saber se ele é negativo ou positivo.
          \item \textbf{O que é \textit{overflow}? O que é \textit{underflow}? Como estes fenômenos podem inferir na vida de um programador de jogos }
          \begin{itemize}
            \item \textbf{overflow} \linebreak
            Dado um sistema de representação numérica, o overflow acontece quando tentamos representar um número maior do que o maior número que esse sistema consegue representar.
            \item \textbf{underflow} \linebreak
            Dado um sistema de representação numérica, o underflow acontece quando tentamos representar um número menor do que o menor número que esse sistema consegue representar.
            \item \textbf{Como esses fenômenos podem interferir na vida de um programador de jogos?} \linebreak
            Imagine em um jogo de luta, onde existem combos de luta com multiplicadores de dano, caso o combo seja tão extraordinário que supere o limite máximo de representação do sistema numérico usado, o dano não será poderá ser representado.
          \end{itemize}
        \end{enumerate}
      \end{enumerate}
\end{document}


% 0 1 2 3  4  5  6    7    8
% 1 2 4 8 16 32 64  128  256

\documentclass{article}
\usepackage{hyperref}
\usepackage[shortlabels]{enumitem}
\usepackage{geometry}
\geometry{a4paper, left=3cm, right=2cm, top=3cm,
bottom=2cm}

\begin{document}
  \begin{enumerate}
    \item \textbf{O que são conjuntos disjuntos} \newline
    São conjuntos que não possuem elementos em comum.
    \item \textbf{Defina as seguintes operações clássicas usando apenas notação matemática}
    \begin{enumerate}[(a)]
      \item \textbf{União de conjuntos:}
      \begin{math}
        \\ A \cup B = x  \ | \ x \in A  \ ou \ x  \in B \\
        Exemplo: \\
        A = \{ a,b,c \} \\
        B = \{ c,d,e \} \\
        A \cup B = \{a,b,c,d,e,f\}
      \end{math}
      \item \textbf{Interseção de conjuntos:}
      \begin{math}
        \\ A \cap B = x \ | \ x \in A \ e \ x \in B \\
        Exemplo: \\
        A = \{ a,b,c \} \\
        B = \{ c,d,e \} \\
        A \cap B = \{c\}
      \end{math}
      \item \textbf{Produto cartesiano:}
      \begin{math}
        \\ A \times B =  (x,y) \ | \ x \in A \ e \ y \in B \\
        Exemplo: \\
        A = \{ a,b,c \} \\
        B = \{ d,e \} \\
        A \times B = \{ (a,d),(a,e),(b,d),(b,e),(c,d),(c,e)\}
      \end{math}
    \end{enumerate}
    \item \textbf{Defina o conjunto das partes de A denotado por $\wp(A)$, depois escreva  $ \wp(A) $ quando:}
    \begin{itemize}
      \item O conjunto das partes é o conjunto definido por:
      \begin{math}
        \wp(A) = \{ X \ | \ X \subset A \}
      \end{math}
    \end{itemize}
    \begin{enumerate}[(a)]
      \item \textbf{A = \{a,b\}:}
      \begin{math}
        \\ \wp(A) = \{\emptyset, \{a\},\{b\},\{a,b\}\}
      \end{math}
      \item \textbf{A = \{1,2,3\}}:
      \begin{math}
        \\ \wp(A)= \{\emptyset, \{1\}, \{2\}, \{3\}, \{1,2\}, \{1,3\}, \{2,3\} \}
      \end{math}
    \end{enumerate}
  \end{enumerate}
\end{document}
